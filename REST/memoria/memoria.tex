\documentclass[12pt,a4paper, spanish]{article}
\usepackage[spanish]{babel}
\usepackage[utf8]{inputenc}
\usepackage{setspace}
\usepackage[
  pdftex,
  pdfauthor={--- Arturo Vidal Peña & Javier Melero Deza ---},
  pdftitle={--- Práctica RESTful 2018/2019 ---},
  hidelinks]{hyperref}

% --- IMAGES ---
\usepackage[pdftex]{graphicx}
\usepackage{subfig}
\usepackage{graphicx}
\usepackage[usenames,dvipsnames]{color}
\DeclareGraphicsExtensions{.png,.jpg,.pdf,.mps,.gif,.bmp}
% --- IMAGES ---

% --- MARGIN DIMENSIONS ---
\frenchspacing \addtolength{\hoffset}{-1.5cm}
\addtolength{\textwidth}{3cm} \addtolength{\voffset}{-2.5cm}
\addtolength{\textheight}{4cm}
\setlength{\headheight}{15pt}
% --- MARGIN DIMENSIONS ---

% --- TITLE DATA ---
\title{\textbf{Práctica RESTful} \\
       \textsc{Sistemas Orientados a Servicios} \\
       \emph{DLSIIS}}
\author{\emph{Vidal Peña, Arturo}\\
        \emph{Melero Deza, Javier}}
\date{\underline{\today}}
% --- TITLE DATA ---

% --- TABLE OF CONTENTS' DOTS
\usepackage{tocloft}
\renewcommand{\cftsecleader}{\cftdotfill{\cftdotsep}}
% --- TABLE OF CONTENTS' DOTS

% --- DOCUMENT ---
\begin{document}

% --- TITLE ---
\maketitle
\thispagestyle{empty}
\pagenumbering{gobble}
\renewcommand*\contentsname{Índice de contenidos}
\tableofcontents
\pagebreak
% --- TITLE ---

\pagenumbering{arabic}

\section{Resumen del diseño del servicio}
Tablas de 'Swagger Editor' y diseño de la persistencia de los datos del servicio.

\newpage
\section{Capturas de la ejecución en un cliente REST (Postman)}
Hay que poner detalles tanto de la invocación como del resultado de la operación.

\newpage
\section{Capturas de la ejecución del cliente de prueba}

\end{document}
% --- DOCUMENT ---
