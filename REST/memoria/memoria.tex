\documentclass[12pt,a4paper, spanish]{article}
\usepackage[spanish]{babel}
\usepackage[utf8]{inputenc}
\usepackage{setspace}
\usepackage[
  pdftex,
  pdfauthor={--- Arturo Vidal Peña & Javier Melero Deza ---},
  pdftitle={--- Práctica RESTful 2018/2019 ---},
  hidelinks]{hyperref}

% --- IMAGES ---
\usepackage[pdftex]{graphicx}
\usepackage{subfigure}
\usepackage{graphicx}
\usepackage{float}
\usepackage[usenames,dvipsnames]{color}
\DeclareGraphicsExtensions{.png,.jpg,.pdf,.mps,.gif,.bmp}
% --- IMAGES ---

% --- TABLES ---
\usepackage{multirow}
% --- TABLES ---

% --- CODE ---
\usepackage{listings}
\definecolor{backcolour}{RGB}{45,45,45}
\definecolor{keycolour}{RGB}{242, 119, 122}
\definecolor{valuecolour}{RGB}{204,204,204}
\definecolor{stringcolour}{RGB}{153,204,153}
\definecolor{integercolour}{RGB}{249,145,87}

\definecolor{codegray}{rgb}{0.5,0.5,0.5}

\newcommand\YAMLcolonstyle{\color{keycolour}\mdseries}
\newcommand\YAMLkeystyle{\color{keycolour}\bfseries}
\newcommand\YAMLvaluestyle{\color{valuecolour}\mdseries}
\newcommand\YAMLstringstyle{\color{stringcolour}\mdseries}
\newcommand\YAMLintegerstyle{\color{integercolour}\mdseries}

\makeatletter

% here is a macro expanding to the name of the language
% (handy if you decide to change it further down the road)
\newcommand\language@yaml{yaml}

\expandafter\expandafter\expandafter\lstdefinelanguage
\expandafter{\language@yaml}
{
	numbers=left,                    
	numbersep=5pt, 
	numberstyle=\tiny\color{codegray},
	backgroundcolor=\color{backcolour},
	breaklines=true,
	keywords={true, false, null,y,n},
	keywordstyle=\color{integercolour}\bfseries,
	basicstyle=\YAMLkeystyle,                     % assuming a key comes first
	sensitive=false,
	captionpos=b,
	comment=[l]{\#},
	morecomment=[s]{/*}{*/},
	commentstyle=\color{purple}\ttfamily,
	stringstyle=\YAMLstringstyle\ttfamily,
	moredelim=[l][\color{orange}]{\&},
	moredelim=[l][\color{magenta}]{*},
	moredelim=**[il][\YAMLcolonstyle{:}\YAMLvaluestyle]{:},   % switch to value style at :
	morestring=[b]',
	morestring=[b]",
	literate =    {---}{{\ProcessThreeDashes}}3
	{>}{{\textcolor{red}\textgreater}}1     
	{|}{{\textcolor{red}\textbar}}1 
	{\ -\ }{{\mdseries\ -\ }}3,
	showstringspaces=false,
}

% switch to key style at EOL
\lst@AddToHook{EveryLine}{\ifx\lst@language\language@yaml\YAMLkeystyle\fi}
\makeatother

\newcommand\ProcessThreeDashes{\llap{\color{cyan}\mdseries-{-}-}}
% --- CODE ---

% --- MARGIN DIMENSIONS ---
\frenchspacing \addtolength{\hoffset}{-1.5cm}
\addtolength{\textwidth}{3cm} \addtolength{\voffset}{-2.5cm}
\addtolength{\textheight}{4cm}
\setlength{\headheight}{15pt}
% --- MARGIN DIMENSIONS ---

% --- TITLE DATA ---
\title{\textbf{Práctica RESTful} \\
       \textsc{Sistemas Orientados a Servicios} \\
       \emph{DLSIIS}}
\author{\emph{Vidal Peña, Arturo}\\
        \emph{Melero Deza, Javier}}
\date{\underline{\today}}
% --- TITLE DATA ---

% --- TABLE OF CONTENTS' DOTS
\usepackage{tocloft}
\renewcommand{\cftsecleader}{\cftdotfill{\cftdotsep}}
% --- TABLE OF CONTENTS' DOTS

% --- DOCUMENT ---
\begin{document}

% --- TITLE ---
\maketitle
\thispagestyle{empty}
\pagenumbering{gobble}
\renewcommand*\contentsname{Índice de contenidos}
\tableofcontents
\pagebreak
% --- TITLE ---

\pagenumbering{arabic}

\section{Diseño de la base de datos}
El sistema constará de tres partes, el cliente Postman, una base de datos y el servidor. La base de datos se compone de 4 tablas: 
\begin{itemize}
	\item Usuarios
	\item Mensajes
	\item Privados
	\item Amistad
\end{itemize}

Dichas tablas se conectan de la siguiente manera:
\begin{figure}[H]
	\centering
	\includegraphics[width=\textwidth]{images/bbdd.png}
\end{figure}
Donde en \textbf{Privados}, \textit{destinatario} es clave foránea, en \textbf{Mensajes} \textit{id\_usuario} es clave foránea y en \textbf{Amistad} tanto \textit{id\_amistad} como \textit{id\_amigo} son claves dependientes de \textit{id} en \textbf{Usuarios} para determinar la relación. Una vez una parte añade a otra como amigo, se añaden los ids de ambos a la tabla \textbf{Amistad}, e independientemente de cual sea su orden, el sistema esta programado para reconocer una amistad de forma recíproca sin repetir entradas.\\

En cuanto al código, esta estructurado en tres partes, el recurso principal de usuario, los “servicios” que son clases que envuelven listas de objetos con \textit{URL} de dichos objetos, y los modelos, los cuales son enviados y recibidos en formato \textit{JSON} principalmente a base de serializar sus atributos.\\

\textbf{\underline{Models}}:
\begin{itemize}
	\item LinkMessage – Estructura de datos que guarda url a mensaje.
	\item LinkPrivate – Estructura de datos que guarda url a mensaje privado.
	\item LinkUser – Estructura de datos que guarda url a usuario.
	\item Message – Mensaje.
	\item MobileData – Estructura necesaria para dar salida a requerimiento de aplicación móvil.
	\item Private – Mensaje privado.
	\item User – Usuario 
\end{itemize} 

\textbf{\underline{Services}}:
\begin{itemize}
	\item message\_service – Estructura que envuelve una lista de urls a mensajes
	\item private\_service – Estructura que envuelve una lista de urls a privados
	\item user\_service – Estructura que envuelve una lista de urls a usuarios
\end{itemize}

Se ha intentado seguir un diseño de \textit{URI} consistente basado en una transición de un contenedor más externo a uno más interno contenido en este último.\\

A la hora de dar salida a los datos, al retornar listas se ha elegido retornar estructuras que contienen referencias y urls a los objetos de la lista en cuestión.
Se ha intentado programar un cliente pero debido a motivos desconocidos el servidor no detectaba determinadas Request (a pesar de que en Postman si funciona) y por tanto el cliente entraba en interbloqueo.\\

Lo que se ha hecho se ha dejado comentado (el paquete se encuentra dentro del propio proyecto).\\

Todas las funcionalidades y requerimientos han sido probados en Postman.

\newpage
\section{Resumen del diseño del servicio}
	\begin{figure}[H]
		\centering
		\includegraphics[width=\textwidth]{images/yaml/index.png}
	\end{figure}
\subsection{Métodos de usuarios}
\subsubsection{Crear un usuario}
\begin{figure}[H]
	\centering
	\includegraphics[width=\textwidth]{images/yaml/userOps/post.png}
\end{figure}
\subsubsection{Obtener info del usuario}
\begin{figure}[H]
	\centering
	\includegraphics[width=\textwidth]{images/yaml/userOps/getUser.png}
\end{figure}
\subsubsection{Obtener todos los usuarios}
\begin{figure}[H]
	\centering
	\includegraphics[width=\textwidth]{images/yaml/userOps/get.png}
\end{figure}
\subsubsection{Modificar un usuario}
\begin{figure}[H]
	\centering
	\includegraphics[width=\textwidth]{images/yaml/userOps/put.png}
\end{figure}
\subsubsection{Eliminar un usuario}
\begin{figure}[H]
	\centering
	\includegraphics[width=\textwidth]{images/yaml/userOps/delete.png}
\end{figure}

\newpage
\subsection{Métodos de amistad}
\subsubsection{Crear una amistad}
\begin{figure}[H]
	\centering
	\includegraphics[width=\textwidth]{images/yaml/userFriendOps/post.png}
\end{figure}
\subsubsection{Obtener amigos de un usuario}
\begin{figure}[H]
	\centering
	\includegraphics[width=\textwidth]{images/yaml/userFriendOps/get.png}
\end{figure}

\newpage
\subsection{Métodos de mensajes}
\subsubsection{Crear un mensaje}
\begin{figure}[H]
	\centering
	\includegraphics[width=\textwidth]{images/yaml/msgOps/post.png}
\end{figure}
\subsubsection{Obtener info del mensaje}
\begin{figure}[H]
	\centering
	\includegraphics[width=\textwidth]{images/yaml/msgOps/getMsg.png}
\end{figure}
\subsubsection{Obtener todos los mensajes de un usuario}
\begin{figure}[H]
	\centering
	\includegraphics[width=\textwidth]{images/yaml/msgOps/get.png}
\end{figure}
\subsubsection{Modificar un mensaje}
\begin{figure}[H]
	\centering
	\includegraphics[width=\textwidth]{images/yaml/msgOps/put.png}
\end{figure}
\subsubsection{Eliminar un mensaje}
\begin{figure}[H]
	\centering
	\includegraphics[width=\textwidth]{images/yaml/msgOps/delete.png}
\end{figure}

\newpage
\subsection{Métodos de mensajes privados}
\subsubsection{Crear un mensaje privado}
\begin{figure}[H]
	\centering
	\includegraphics[width=\textwidth]{images/yaml/privateMsgOps/post.png}
\end{figure}
\subsubsection{Obtener info del mensaje privado}
\begin{figure}[H]
	\centering
	\includegraphics[width=\textwidth]{images/yaml/privateMsgOps/getMsg.png}
\end{figure}
\subsubsection{Obtener todos los mensajes privados de un usuario}
\begin{figure}[H]
	\centering
	\includegraphics[width=\textwidth]{images/yaml/privateMsgOps/get.png}
\end{figure}

\newpage
\subsection{Fichero YAML del servicio}
El siguiente código se ha realizado usando la herramienta Swagger Editor. El fichero se encuentra adjunto a este documento y las capturas previas. 
\lstinputlisting[language=YAML]{yaml/config.yaml}

\newpage
\subsubsection{Definición de los métodos de usuario}
\lstinputlisting[language=YAML]{yaml/usuarios.yaml}

\newpage
\subsubsection{Definición de los métodos de mensajes}
\lstinputlisting[language=YAML]{yaml/mensajes.yaml}

\newpage
\subsubsection{Definición de los métodos de mensajes privados}
\lstinputlisting[language=YAML]{yaml/privados.yaml}

\newpage
\subsubsection{Definición de los métodos de amistades}
\lstinputlisting[language=YAML]{yaml/amigos.yaml}

\newpage
\subsubsection{Definición de los esquemas}
\lstinputlisting[language=YAML]{yaml/components.yaml}

\newpage
\section{Capturas de la ejecución en el cliente Postman}
Hay que poner detalles tanto de la invocación como del resultado de la operación.

\begin{figure}[H]
	\centering
	\includegraphics[width=0.75\textwidth]{images/final/1.png}
	\caption{Se añade el usuario 1 de nombre Rafa.}
\end{figure}
\begin{figure}[H]
	\centering
	\includegraphics[width=0.75\textwidth]{images/final/2.png}
	\caption{Se añade el usuario 2 de nombre Javi.}
\end{figure}
\begin{figure}[H]
	\centering
	\includegraphics[width=0.75\textwidth]{images/final/3.png}
	\caption{Se añade el usuario 3 de nombre Laura.}
\end{figure}
\begin{figure}[H]
	\centering
	\includegraphics[width=0.75\textwidth]{images/final/4.png}
	\caption{Se añade el usuario 4 de nombre Jordi.}
\end{figure}
\begin{figure}[H]
	\centering
	\includegraphics[width=0.75\textwidth]{images/final/5.png}
	\caption{Se obtienen todos los usuarios existentes.}
\end{figure}
\begin{figure}[H]
	\centering
	\includegraphics[width=0.75\textwidth]{images/final/6.png}
	\caption{Se obtiene la información del usuario 1 (Rafa).}
\end{figure}
\begin{figure}[H]
	\centering
	\includegraphics[width=0.75\textwidth]{images/final/7.png}
	\caption{Se modifica parte de la información del usuario 1 (Rafa).}
\end{figure}
\begin{figure}[H]
	\centering
	\includegraphics[width=0.75\textwidth]{images/final/8.png}
	\caption{Se obtiene la información del usuario 1 (Rafa).}
\end{figure}
\begin{figure}[H]
	\centering
	\includegraphics[width=0.75\textwidth]{images/final/9.png}
	\caption{Se elimina el usuario 4 (Jordi).}
\end{figure}
\begin{figure}[H]
	\centering
	\includegraphics[width=0.75\textwidth]{images/final/10.png}
	\caption{Se obtienen todos los usuarios existentes.}
\end{figure}
\begin{figure}[H]
	\centering
	\includegraphics[width=0.75\textwidth]{images/final/11.png}
	\caption{El usuario 1 (Rafa) crea un mensaje (1).}
\end{figure}
\begin{figure}[H]
	\centering
	\includegraphics[width=0.75\textwidth]{images/final/12.png}
	\caption{El usuario 2 (Javi) crea un mensaje (2).}
\end{figure}
\begin{figure}[H]
	\centering
	\includegraphics[width=0.75\textwidth]{images/final/13.png}
	\caption{El usuario 3 (Laura) crea un mensaje (3).}
\end{figure}
\begin{figure}[H]
	\centering
	\includegraphics[width=0.75\textwidth]{images/final/14.png}
	\caption{El usuario 3 (Laura) crea un mensaje (4).}
\end{figure}
\begin{figure}[H]
	\centering
	\includegraphics[width=0.75\textwidth]{images/final/15.png}
	\caption{Se obtienen los mensajes creados por el usuario 3 (Laura).}
\end{figure}
\begin{figure}[H]
	\centering
	\includegraphics[width=0.75\textwidth]{images/final/16.png}
	\caption{El usuario 3 (Laura) modifica el mensaje 3.}
\end{figure}
\begin{figure}[H]
	\centering
	\includegraphics[width=0.75\textwidth]{images/final/17.png}
	\caption{Se obtiene la información del mensaje 3.}
\end{figure}
\begin{figure}[H]
	\centering
	\includegraphics[width=0.75\textwidth]{images/final/18.png}
	\caption{Se elimina el mensaje 3 del usuario 3 (Laura).}
\end{figure}
\begin{figure}[H]
	\centering
	\includegraphics[width=0.75\textwidth]{images/final/19.png}
	\caption{Se obtienen los mensajes creados por el usuario 3 (Laura).}
\end{figure}
\begin{figure}[H]
	\centering
	\includegraphics[width=0.75\textwidth]{images/final/20.png}
	\caption{El usuario 3 (Laura) crea un mensaje privado para el usuario 2 (Javi).}
\end{figure}
\begin{figure}[H]
	\centering
	\includegraphics[width=0.75\textwidth]{images/final/21.png}
	\caption{Se obtienen los mensajes privados del usuario 2 (Javi).}
\end{figure}
\begin{figure}[H]
	\centering
	\includegraphics[width=0.75\textwidth]{images/final/22.png}
	\caption{Se obtiene la información del mensaje privado 1 del usuario 2 (Javi).}
\end{figure}
\begin{figure}[H]
	\centering
	\includegraphics[width=0.75\textwidth]{images/final/23.png}
	\caption{Se añade al usuario 2 (Javi) como amigo del usuario 1 (Rafa).}
\end{figure}
\begin{figure}[H]
	\centering
	\includegraphics[width=0.75\textwidth]{images/final/24.png}
	\caption{Se añade al usuario 3 (Laura) como amigo del usuario 1 (Rafa).}
\end{figure}
\begin{figure}[H]
	\centering
	\includegraphics[width=0.75\textwidth]{images/final/25.png}
	\caption{Se obtienen los amigos del usuario 1 (Rafa).}
\end{figure}
\begin{figure}[H]
	\centering
	\includegraphics[width=0.75\textwidth]{images/final/26.png}
	\caption{Se obtienen los amigos del usuario 2 (Javi).}
\end{figure}
\begin{figure}[H]
	\centering
	\includegraphics[width=0.75\textwidth]{images/final/27.png}
	\caption{Se elimina al usuario 1 (Rafa) como amigo del usuario 2 (Javi).}
\end{figure}
\begin{figure}[H]
	\centering
	\includegraphics[width=0.75\textwidth]{images/final/28.png}
	\caption{Se obtienen los amigos del usuario 1 (Rafa).}
\end{figure}
\begin{figure}[H]
	\centering
	\includegraphics[width=0.75\textwidth]{images/final/29.png}
	\caption{Se añade el usuario 1 (Rafa) como amigo del usuario 2 (Javi).}
\end{figure}

\newpage

\subsection{Obtención de mensajes similar a FaceBook}
\begin{figure}[H]
	\centering
	\includegraphics[width=0.75\textwidth]{images/final/fb1.png}
	\caption{Obtención de los amigos del usuario 1.}
\end{figure}

\begin{figure}[H]
	\centering
	\includegraphics[width=0.75\textwidth]{images/final/fb2.png}
	\caption{Creación de un mensaje del usuario 2.}
\end{figure}

\begin{figure}[H]
	\centering
	\includegraphics[width=0.75\textwidth]{images/final/fb3.png}
	\caption{Obtención de los mensajes de los amigos del usuario 1.}
\end{figure}

\subsection{Aplicación móvil}
\begin{figure}[H]
	\centering
	\includegraphics[width=0.75\textwidth]{images/final/movil.png}
	\caption{Obtención de la información del usuario 1 simulando una aplicación móvil.}
\end{figure}
\newpage
\section{ANEXO: Problemas encontrados}

Los fallos encontrados referentes a la Máquina virtual son:

\begin{itemize}
	\item Con la aplicación Postman, al probar los métodos HTTP.
	\item Con la aplicación Eclipse, que algunas veces no encontraba las URIs implementadas.
	\item Con la baja calidad de las capturas de pantalla del SO utilizado (probablemente por la antigüedad del mismo).
\end{itemize} 

\begin{figure}[H]
	\centering
	\subfigure{\includegraphics[width=0.45\textwidth]{images_old/error1.jpg}}
	\subfigure{\includegraphics[width=0.45\textwidth]{images_old/error2.jpg}}
	\caption{Fallo de la aplicación Postman al probar métodos.}
\end{figure}



\end{document}
% --- DOCUMENT ---
