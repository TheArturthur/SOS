\documentclass[12pt,a4paper, spanish]{article}
\usepackage[spanish]{babel}
\usepackage[utf8]{inputenc}
\usepackage{setspace}
\usepackage[
  pdftex,
  pdfauthor={--- Arturo Vidal Peña & Javier Melero Deza ---},
  pdftitle={--- Práctica RESTful 2018/2019 ---},
  hidelinks]{hyperref}

% --- IMAGES ---
\usepackage[pdftex]{graphicx}
\usepackage{subfig}
\usepackage{graphicx}
\usepackage{float}
\usepackage[usenames,dvipsnames]{color}
\DeclareGraphicsExtensions{.png,.jpg,.pdf,.mps,.gif,.bmp}
% --- IMAGES ---

% --- MARGIN DIMENSIONS ---
\frenchspacing \addtolength{\hoffset}{-1.5cm}
\addtolength{\textwidth}{3cm} \addtolength{\voffset}{-2.5cm}
\addtolength{\textheight}{4cm}
\setlength{\headheight}{15pt}
% --- MARGIN DIMENSIONS ---

% --- TITLE DATA ---
\title{\textbf{Práctica RESTful} \\
       \textsc{Sistemas Orientados a Servicios} \\
       \emph{DLSIIS}}
\author{\emph{Vidal Peña, Arturo}\\
        \emph{Melero Deza, Javier}}
\date{\underline{\today}}
% --- TITLE DATA ---

% --- TABLE OF CONTENTS' DOTS
\usepackage{tocloft}
\renewcommand{\cftsecleader}{\cftdotfill{\cftdotsep}}
% --- TABLE OF CONTENTS' DOTS

% --- DOCUMENT ---
\begin{document}

% --- TITLE ---
\maketitle
\thispagestyle{empty}
\pagenumbering{gobble}
\renewcommand*\contentsname{Índice de contenidos}
\tableofcontents
\pagebreak
% --- TITLE ---

\pagenumbering{arabic}

\section{Resumen del diseño del servicio}
Tablas de 'Swagger Editor' y diseño de la persistencia de los datos del servicio.

\newpage
\section{Capturas de la ejecución en un cliente REST (Postman)}
Hay que poner detalles tanto de la invocación como del resultado de la operación.

\begin{figure}[H]
	\centering
	\includegraphics[width=0.75\textwidth]{images/captura1.jpg}
	\caption{Método POST para insertar un amigo en el usuario con id 2.}
\end{figure}

\begin{figure}[H]
	\centering
	\includegraphics[width=0.75\textwidth]{images/captura2.jpg}
	\caption{Método DELETE para borrar un mensaje del usuario con id 1.}
\end{figure}

\begin{figure}[H]
	\centering
	\includegraphics[width=0.75\textwidth]{images/captura3.jpg}
	\caption{Método GET para obtener la información del usuario con id 1.}
\end{figure}

\begin{figure}[H]
	\centering
	\includegraphics[width=0.75\textwidth]{images/captura4.jpg}
	\caption{Método GET para obtener la información de todos los usuarios.}
\end{figure}

%\begin{figure}[H]
%	\centering
%	\includegraphics[width=0.75\textwidth]{images/captura5.jpg}
%	\caption{Método GET}
%\end{figure}

\begin{figure}[H]
	\centering
	\includegraphics[width=0.75\textwidth]{images/captura6.jpg}
	\caption{Método GET para obtener los amigos del usuario con id 2.}
\end{figure}

\begin{figure}[H]
	\centering
	\includegraphics[width=0.75\textwidth]{images/captura7.jpg}
	\caption{Método GET para obtener la información del mensaje con id 1 del usuario con id 1.}
\end{figure}

\begin{figure}[H]
	\centering
	\includegraphics[width=0.75\textwidth]{images/captura8.jpg}
	\caption{Método GET para obtener los mensajes del usuario con id 1.}
\end{figure}

\begin{figure}[H]
	\centering
	\includegraphics[width=0.75\textwidth]{images/captura9.jpg}
	\caption{Método POST para añadir un nuevo usuario con id 3.}
\end{figure}

\begin{figure}[H]
	\centering
	\includegraphics[width=0.75\textwidth]{images/captura10.jpg}
	\caption{Método POST para añadir un nuevo usuario con id 1.}
\end{figure}

\begin{figure}[H]
	\centering
	\includegraphics[width=0.75\textwidth]{images/captura11.jpg}
	\caption{Método POST para añadir un nuevo mensaje privado desde el usuario con id 2 al usuario con id 3.}
\end{figure}

\begin{figure}[H]
	\centering
	\includegraphics[width=0.75\textwidth]{images/captura12.jpg}
	\caption{Método PUT para modificar datos pertenecientes al usuario con id 1.}
\end{figure}

\begin{figure}[H]
	\centering
	\includegraphics[width=0.75\textwidth]{images/captura13.jpg}
	\caption{Método PUT para modificar datos pertenecientes al usuario con id 1.}
\end{figure}

\begin{figure}[H]
	\centering
	\includegraphics[width=0.75\textwidth]{images/captura14.jpg}
	\caption{Método GET para obtener la información del usuario con id 1.}
\end{figure}

\begin{figure}[H]
	\centering
	\includegraphics[width=0.75\textwidth]{images/captura15.jpg}
	\caption{Método GET para obtener los usuarios cuyo nombre empieza por "Ja".}
\end{figure}

\begin{figure}[H]
	\centering
	\includegraphics[width=0.75\textwidth]{images/captura16.jpg}
	\caption{Método PUT para modificar el mensaje con id 2 de la página del usuario con id 1.}
\end{figure}

\begin{figure}[H]
	\centering
	\includegraphics[width=0.75\textwidth]{images/captura17.jpg}
	\caption{Método PUT para modificar el mensaje con id 2 de la página del usuario con id 1.}
\end{figure}

\begin{figure}[H]
	\centering
	\includegraphics[width=0.75\textwidth]{images/captura18.jpg}
	\caption{Método DELETE para eliminar al usuario con id 3 de la lista de amigos del usuario con id 2.}
\end{figure}

%\begin{figure}[H]
%	\centering
%	\includegraphics[width=0.75\textwidth]{images/captura19.jpg}
%	\caption{Método DELETE}
%\end{figure}

\begin{figure}[H]
	\centering
	\includegraphics[width=0.75\textwidth]{images/captura20.jpg}
	\caption{Método GET para obtener la lista de amigos del usuario con id 2.}
\end{figure}

\begin{figure}[H]
	\centering
	\includegraphics[width=0.75\textwidth]{images/captura21.jpg}
	\caption{Método POST para añadir un mensaje en la página del usuario con id 1.}
\end{figure}

\begin{figure}[H]
	\centering
	\includegraphics[width=0.75\textwidth]{images/captura22.jpg}
	\caption{Método POST para añadir un mensaje en la página del usuario con id 3.}
\end{figure}

\begin{figure}[H]
	\centering
	\includegraphics[width=0.75\textwidth]{images/captura23.jpg}
	\caption{Método GET para obtener los mensajes de amigos del usuario con id 2.}
\end{figure}

\begin{figure}[H]
	\centering
	\includegraphics[width=0.75\textwidth]{images/captura24.jpg}
	\caption{Método GET para obtener el mensaje con id 3 del usuario con id 1.}
\end{figure}

\begin{figure}[H]
	\centering
	\includegraphics[width=0.75\textwidth]{images/captura25.jpg}
	\caption{Método GET para obtener el mensaje con id 4 del usuario con id 3.}
\end{figure}

\end{document}
% --- DOCUMENT ---
